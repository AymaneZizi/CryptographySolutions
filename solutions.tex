\documentclass{article}

\usepackage{hyperref}
\hypersetup{pdfauthor={Cristian Adrián Ontivero}}
\usepackage[utf8]{inputenc}
\usepackage{graphicx}
\usepackage{gensymb} % for the degree symbol
\usepackage{caption}
\usepackage{float}
\usepackage{color}
\usepackage{mathtools}
\usepackage{enumitem}
\usepackage{calc}
\usepackage[hang,flushmargin]{footmisc} 

% For the commutative diagrams.
\usepackage{tikz-cd}

\usepackage{tikz}
\usetikzlibrary{automata, positioning, arrows,
shapes,
fit, % for the dashed boxes on Thompson's construction
calc
}
\tikzset{%
  node distance=3cm, % specifies the minimum distance between two nodes. Change if necessary.
  every state/.style={thick}, % sets the properties for each ’state’ node
  double distance=2.5pt,
  shorten >= 2pt, shorten <= 2pt,
  initial text=$ $,
  every edge/.style={%
    draw,->, >=stealth, auto, semithick
  }
}
\graphicspath{{imgs/}}

\definecolor{darkblue}{RGB}{49,130,189}

\newlength\tindent
\setlength{\tindent}{\parindent}
\setlength{\parindent}{0pt}
\renewcommand{\indent}{\hspace*{\tindent}}

%These tell TeX which packages to use.
\usepackage{array,epsfig}
\usepackage{amsmath}
\usepackage{amsfonts}
\usepackage{amssymb}
\usepackage{amsxtra}
\usepackage{amsthm}
\usepackage{mathrsfs}

%Here I define some theorem styles and shortcut commands for symbols I use often
\theoremstyle{definition}

\newcommand{\lra}{\longrightarrow}
\newcommand{\ra}{\rightarrow}
\newcommand{\surj}{\twoheadrightarrow}
\newcommand{\graph}{\mathrm{graph}}
\newcommand{\bb}[1]{\mathbb{#1}}
\newcommand{\Z}{\bb{Z}}
\newcommand{\Q}{\bb{Q}}
\newcommand{\R}{\bb{R}}
\newcommand{\C}{\bb{C}}
\newcommand{\N}{\bb{N}}
\newcommand{\M}{\mathbf{M}}
\newcommand{\m}{\mathbf{m}}
\newcommand{\MM}{\mathscr{M}}
\newcommand{\HH}{\mathscr{H}}
\newcommand{\Om}{\Omega}
\newcommand{\Ho}{\in\HH(\Om)}
\newcommand{\bd}{\partial}
\newcommand{\del}{\partial}
\newcommand{\bardel}{\overline\partial}
\newcommand{\textdf}[1]{\textbf{\textsf{#1}}\index{#1}}
\newcommand{\ip}[2]{\left\langle{#1},{#2}\right\rangle}
\newcommand{\inter}[1]{\mathrm{int}{#1}}
\newcommand{\exter}[1]{\mathrm{ext}{#1}}
\newcommand{\cl}[1]{\mathrm{cl}{#1}}
\newcommand{\ds}{\displaystyle}
\newcommand{\vol}{\mathrm{vol}}
\newcommand{\cnt}{\mathrm{ct}}
\newcommand{\osc}{\mathrm{osc}}
\newcommand{\LL}{\mathbf{L}}
\newcommand{\UU}{\mathbf{U}}
\newcommand{\support}{\mathrm{support}}
\newcommand{\AND}{\;\wedge\;}
\newcommand{\OR}{\;\vee\;}
\newcommand{\Oset}{\varnothing}
\newcommand{\st}{\ni}
\newcommand{\wh}{\widehat}

\newcommand{\PS}{\mathcal{P}}
\newcommand{\set}[1]{\left\{#1\right\}}
\newcommand{\bra}[1]{\left[#1\right]}
\newcommand{\abs}[1]{\left|#1\right|}
\newcommand{\paren}[1]{\left(#1\right)}
\newcommand{\ec}[1]{{\left[#1\right]}_{\sim}}
\newcommand{\en}{\mathbin{\rotatebox[origin=c]{90}{\scriptsize $\circlearrowright$}}}
\newcommand{\id}{\mathrm{id}}
\newcommand{\cod}{\mathrm{cod~}}
\newcommand{\dom}{\mathrm{dom~}}
\newcommand{\im}{\mathrm{im~}}
\newcommand{\thra}{\twoheadrightarrow}
\newcommand{\emptystr}{\varepsilon}
\newcommand{\emptylan}{\emptyset}
\newcommand{\pdv}[2]{\partial_{#1} \bigl(#2\bigr)}

% Crypto commands
\newcommand{\Gen}{\mathsf{Gen}}
\newcommand{\Enc}{\mathsf{Enc}}
\newcommand{\Dec}{\mathsf{Dec}}

\newcommand{\Ms}{\mathcal{M}} % Message space
\newcommand{\Ks}{\mathcal{K}} % Key space
\newcommand{\Cs}{\mathcal{C}} % Ciphertext space
\newcommand{\Adv}{\mathcal{A}} % Adversary
\newcommand{\negl}{\mathrm{negl}} % Adversary
\newcommand{\priveav}{\text{PrivK}_{\mathcal{A},\Pi}^\text{eav}}

%Pagination stuff.
\setlength{\topmargin}{-.3 in}
\setlength{\oddsidemargin}{0in}
\setlength{\evensidemargin}{0in}
\setlength{\textheight}{9.in}
\setlength{\textwidth}{6.5in}
\pagestyle{empty}

% The problem environment is a regular ams theorem environment with "Problem"
% text and some leading space to give some separation between the problems.
\theoremstyle{definition}
\newtheorem{problem-internal}{Problem}[subsection]
\newenvironment{problem}{
  \medskip
  \begin{problem-internal}
}{
  \end{problem-internal}
}

% The solution environment is a proof environment with the "solution" text as
% well as the following adjustments:
% - No indent on paragraphs;
% - A small amount of space between paragraphs.
%
% Note: The negative space at the beginning is to remove the space before the
% first paragraph in the solution.
%\newenvironment{solution}{%
  %\begin{proof}[Solution]
  %\vspace{-8px}
  %\setlength{\parskip}{4px}
  %\setlength{\parindent}{0px}
%}{
  %\end{proof}
%}
\theoremstyle{definition}
\newtheorem{solution-internal}{}[subsection]
\newenvironment{solution}{
  \begin{solution-internal}
}{
  \end{solution-internal}
}

% The chngcntr ("change counter") package is used here so that subsection
% numbers are written without the leading section number. This takes place in
% the subsection headings as well as the theorem environment numbering.
%
% Before:
% 1. Section
% 1.1. Subsection
% Problem 1.1.1. What is 1 + 1?
% Problem 1.1.2. What is 1 + 2?
%
% After:
% 1. Section
% 1. Subsection
% Problem 1.1. What is 1 + 1?
% Problem 1.2. What is 1 + 2?
\usepackage{chngcntr}
\counterwithout{subsection}{section}

% Renewing the \thesection command changes the section numbers to roman
% numerals. This matches the style of the Aluffi textbook.
%
% Before:
% 1. Section
% 1.1. Subsection
%
% After:
% I. Section
% I.1. Subsection
%\renewcommand{\thesection}{\Roman{section}} 
%\renewcommand{\thesubsection}{\Roman{subsection}} 

\begin{document}

\section*{Chapter 1}
\setcounter{section}{1}
\setcounter{subsection}{1}

% Problem 1.1
\begin{solution}
  abcdefghijklmnopqrstuvwxyz =\gg hcjkfeyvbuxzplomtgwqiasdrn

Decryption the ciphertext provided :

CRYPTOGRAPHICSYSTEMSAREEXTREMELYDIFFICULTTOBUILDNEVERTHELESSF
ORSOMEREASONMANYNONEXPERTSINSISTONDESIGNINGNEWENCRYPTIONSCHEM
ESTHATSEEMTOTHEMTOBEMORESECURETHANANYOTHERSCHEMEONEARTHTHEUNF
ORTUNATETRUTHHOWEVERISTHATSUCHSCHEMESAREUSUALLYTRIVIALTOBREAK
\end{solution}

\begin{solution}
  Denoting $\set{0, \dots, 25}$ by $\Z_{26}$, we have:
  \begin{itemize}
    \item $\Gen$: outputs a uniform key $k$ from the set of bijections
      $p\colon\Z_{26} \to \Z_{26}$, where we are associating each letter of the
      English alphabet, in order, with the correspoding number in $\Z_{26}$.
    \item $\Enc$: The encryption of the message $m = m_1\cdots m_{\ell}$, where
      $m_i \in \Z_{26}$ with key $k$ is given by:
      \[ \Enc_k(m_1 \cdots m_{\ell}) = c_1 \cdots c_{\ell}\]
      where $c_i = k(m_i)$, i.e.\ we apply bijection $k$ to $m_i$.
    \item $\Dec$: The decryption of the ciphertext $c = c_1 \cdots c_{\ell}$,
      where $c_i \in \Z_{26}$ with key $k$ is given by:
      \[ \Dec_k(c_1 \cdots c_{\ell}) = k^{-1}(c_1) \cdots k^{-1}(c_{\ell}) = m_1 \cdots
      m_{\ell}\]
      where $k^{-1}$ is the inverse of the function k, which exists because $k$
      is bijective.
  \end{itemize}
\end{solution}

% 1.3
\begin{solution}
  TODO
\end{solution}

% 1.4
\begin{solution}
  TODO
\end{solution}

\begin{solution}
  Shift cipher: the encryption of a single character sufficies, since $c = m +
  k \mod 26$, so $k = c - m \mod 26$.
  
  Monoalphabetic substitution: if the alphabet contains $n$ characters, at least
  $n-1$ distinct characters are necessary to recover the key (as the
  $n^{\text{th}})$ character is determined once the other $n-1$ characters are).
  By choosing $m = m_1 \cdots m_{n - 1}$ with $m_1 \neq m_2 \neq \cdots \neq m_{n-1}$, we have $c = c_1 \cdots c_{n-1} = k(m_1) \cdots k(m_{n-1})$, and we may find bijection k.

  Vigenère: given a key $k$ of length $n$, $n$ characters suffice to recover the
  key, as each part of it can be recovered as in the shift cipher.
\end{solution}
\begin{solution}
  Since the distance that each character is shifted by is fixed, the attacker
  can choose {\tt abcd} if the ciphertext contains consecutive characters (e.g.
  {\tt mnop}) and {\tt bedg} otherwise.
\end{solution}

\begin{solution}
  It is not possible with period 2. With period 3:

  \begin{center}
\setlength{\tabcolsep}{2pt}
  \begin{tabular}{cccc p{5mm} cccc}
    0 & 1 & 2 & 3 & & 1 & 4 & 3 & 6 \\
    a & b & c & d & & b & e & d & g \\
    $k_1$ & $k_2$ & $k_3$ & $k_4$ & & $k_1$ & $k_2$ & $k_3$ & $k_4$ \\
  \end{tabular}
  \end{center}

  %Thus, the message is {\tt abcd} iff $c_4 -c_1 \cong 3 \pmod 26$.
  As for Vigenère with period 4, $\abs{\Ks} = \abs{\Ms}$, hence we have perfect
  secrecy.
\end{solution}

% 1.8
\begin{solution}
  TODO
\end{solution}

\section*{Chapter 2}
\setcounter{section}{2}
\setcounter{subsection}{2}
\setcounter{solution-internal}{0}
\begin{solution}
  TODO
\end{solution}
\begin{solution}
$\Enc$ takes a message $m \in \Ms$ and a key $k \in \Ks$, and is randomised (it
gets a number of bits from some random tape that it uses as input as well).
Instead of implicitly getting the random bits, we make it explicit passing them
as input, by redifining the key space to $\Ks \times \mathcal{R}$ (where
$\mathcal{R}$ is the set of all possible random tapes of the aximal length we
could need):

Thus, $\Enc$ becomes deterministic, as it has all the randomness it needs in the
new-style key.
\end{solution}

% 2.3
\begin{solution}
  Consider a scheme with $1$ bit of plaintext, $3$ bits of key, and $2$ bits of
  ciphertext. The two bits of ciphertext, $c_0$ and $c_1$, are
  obtained as follows:
  \begin{align*}
    c_0 &= m_0 \oplus k_0 \\
    c_1 &= (k_2 \AND k_1) \oplus m_0 \oplus k_0
  \end{align*}
  The possible ciphertexts can be seen in the following table:
  \begin{table}
    \begin{tabular}{r|c|c}
       & $0$ & $1$ \\
      \hline
      $(0,0,0)$ & $(0,0)$ & $(1,1)$ \\
      $(0,0,1)$ & $(0,0)$ & $(1,1)$ \\
      $(0,1,0)$ & $(0,0)$ & $(1,1)$ \\
      $(0,1,1)$ & $(0,1)$ & $(1,0)$ \\
      $(1,0,0)$ & $(1,1)$ & $(0,0)$ \\
      $(1,0,1)$ & $(1,1)$ & $(0,0)$ \\
      $(1,1,0)$ & $(1,1)$ & $(0,0)$ \\
      $(1,1,1)$ & $(1,0)$ & $(0,1)$
    \end{tabular}
  \end{table}

  The scheme is perfectly secure:
  \begin{align*}
    P[M = 0|C = (0,0)] &= \frac{P[M = 0] \cdot \left(P[K = (0,0,0)] + P[K=(0,0,1)] + P[K=(0,1,0)]\right)}{P[C=(0,0)]} \\
                       &= \frac{P[M = 0] \cdot \frac{3}{8}}{\frac{6}{16}} =
    P[M=0] \\[16pt]
    P[M = 0|C = (0,1)] &= \frac{P[M = 0] \cdot P[K = (0,1,1)]}{P[C=(0,1)]}  \\
                       &= \frac{P[M = 0] \cdot \frac{1}{8}}{\frac{2}{16}} =
    P[M=0] \\
    \makebox[\widthof{etc.}]{\vdots} & \\
    etc. &
    \end{align*}
  
    but $\frac{3}{8} = P[C = (0,0)] \neq P[C=(0,1)] = \frac{1}{8}$.
\end{solution}

% 2.4
\begin{solution}
  Assume that the encryption scheme is perfectly secret, and fix messages $m_0,
  m_1 \in \Ms$ and a ciphertext $c \in \Cs$. By Lemma 2.2 of the $1^{\text{st}}$
  edition of the book, we have:
  \[ P[C=c | M = m_0] = P[C = c] = P[C = c | M = m_1] \]
  Completing the proof of the ``only if'' ($\Rightarrow$) direction.
\end{solution}
Note that $P[\Enc_k(m) = c] = P[C=c|M=m]$, as explained in page 30. It is also
worth pointing that Lemma 2.2 is an equivalent formulation of perfect secrecy, stating:

% TODO use some definition environment here
An encryption scheme $(\Gen, \Enc, \Dec)$ over a message space $\Ms$ is
perfectly secret iff for every probability distribution over $\Ms$, every
message $m \in \Ms$, and every ciphertext $c \in \Cs$:
\[ P[C = c | M = m] = P[C = c] \]

\begin{solution}
  We need to prove that an encryption scheme $\Pi$ is perfectly secret iff it is
  perfectly indistinguishable.
  \begin{item}
    \item[$(\Rightarrow):$] In what follows, we make the assumption that the
      adversary is deterministic. Suppose $\Pi$ is perfectly secret. We need
      to show that $P[\priveav = 1] = \frac{1}{2}$.
      \begin{align*}
        P[\priveav = 1] &= P[b' = b] \\
                        &= P[b' = 0|M = m_0] \cdot P[M=m_0] + P[b'=1|M = m_1]\cdot P[M=m_1] \\
                        &= \frac{1}{2} \left( P[b' = 0|M = m_0] + P[b'=1|M = m_1] \right)
      \end{align*}
      Note that in the last step we used that $P[M = m_0] = P[M = m_1] =
      \frac{1}{2}$. Essentially what the adversary does is try to partition the
      ciphertext space $\Cs$ into two subsets $\Cs_0, \Cs_1$ such that $\Cs =
      \Cs_0 \cup \Cs_1$ and $\Cs_0 \cap \Cs_1 = \emptyset$. If the attacker gets
      $c \in \Cs_0$ it outputs 0, else if $c \in \Cs_1$, it outputs 1. We thus
      proceed:
      \begin{align*}
        & \frac{1}{2}\left( P[b'=0|M=m_0] + P[b'=1|M=m_1]\right) \\
        =& \frac{1}{2}\left(\sum_{c \in \Cs_0} P[C=c|M=m_0] + \sum_{c \in \Cs_1} P[C=c|M=m_1]\right) \\
        =& \frac{1}{2}\left(\sum_{c \in \Cs_0} P[C=c] + \sum_{c \in \Cs_1} P[C=c]\right) \\
        =& \frac{1}{2}\left(P[c \in \Cs_0] + P[c \in \Cs_1]\right) = \frac{1}{2}
      \end{align*}
      Note that the second and third lines are equal by an equivalent
      formulation of perfect secrecy, and the last equality holds since $P[c \in
      \Cs_0] + P[c \in \Cs_1] = 1$, because $\Cs_0$ and $\Cs_1$ are mutually
      exclusive and exhaustive.
    \item[$(\Leftarrow):$] We prove the contrapositive, i.e.\ $\neg
      \text{Perfect secrecy} \Rightarrow \neg \text{Adversarial
      indistinguishability}$. Suppose $\Pi$ is not perfectly secret, then
      $\exists m_0',m_1' \in \Ms$ and $c' \in \Cs$ such that:
      \[ P[C=c'|M=m_0'] \neq P[C=c'|M=m_1'] \]
      (Using an equivalent formulation to the original perfect secrecy).
      Let $\Adv$ be an adversary that chooses $m_0'$ and $m_1'$. If it receives
      $c'$, $\Adv$ outputs $b'=0$, otherwise $b' \leftarrow \set{0,1}$ (the
      randomness is to ensure we can separate out the case when $C=c'$).

      \begin{align*}
        P[\priveav = 1] &= P[b=b']\\
                        &= P[b=b'|M=m_0']P[M=m_0'] + P[b=b'|M=m_1']P[M=m_1'] \\
                        &= \frac{1}{2}\left(P[b=b'|M=m_0'] + P[b=b'|M=m_1'] \right) \\
        P[b=b'|M=m_0'] &= P[C=c'|M=m_0']\cdot P[b=b'|M=m_0',C=c']\\
                       &+ P[C\neq c'|M=m_0']\cdot P[b=b'|M=m_0', C\neq c'] \\
        &= P[C=c'|M=m_0']\cdot 1 + P[C\neq c'|M=m_0']\cdot \frac{1}{2} \\
        P[b=b'|M=m_1'] &= P[C=c'|M=m_1']\cdot P[b=b'|M=m_1',C=c']\\
                       &+ P[C\neq c'|M=m_1']\cdot P[b=b'|M=m_1', C\neq c'] \\
        &= P[C=c'|M=m_1']\cdot 0 + P[C\neq c'|M=m_1']\cdot \frac{1}{2} \\
        &= P[C\neq c'|M=m_1']\cdot \frac{1}{2}
      \end{align*}
    Substituting back:
      \begin{align*}
        P[\priveav = 1] &= \frac{1}{2}\left( P[C=c'|M=m_0'] + P[C\neq c'|M=m_0'] \frac{1}{2} + P[C\neq c'|M=m_1'] \frac{1}{2}\right )  \\
                        &= \frac{1}{2} P[C=c'|M=m_0'] + \frac{1}{4}\left(1 - P[C
      = c'|M=m_0']\right) + \frac{1}{4} P[C\neq c'|M=m_1'] \\
                        &= \frac{1}{4} + \frac{1}{4}\left(P[C=c'|M=m_0'] +  P[C\neq c'|M=m_1'] \right) \\
                        &\neq \frac{1}{4} + \frac{1}{4}\left(P[C=c'|M=m_1'] +  P[C\neq c'|M=m_1'] \right) \\
                        &=\frac{1}{2}
      \end{align*}
      The inequality comes from supposing that there is no perfect secrecy.
      Therefore:
      \[ P[\priveav = 1] \neq \frac{1}{2} \]
      Hence $\Pi$ does not have adversarial indistinguishability.
  \end{item}
\end{solution}
\begin{solution}
  $ $ 
  \begin{enumerate}[label=\alph*.]
  \item Take $m = 0$ and $c = 0$ as a counterexample. Then we have:

  \begin{minipage}{.45\textwidth}
  \begin{align*}
    P[C=c | M=m] &= P[\Enc_k(m)=c] \\
                 &= P[m + K \equiv c \pmod 5] \\
                 &= P[K \equiv c -m \pmod 5] \\
                 &= \frac{2}{6} = \frac{1}{3} 
  \end{align*}
  \end{minipage}\hfill%
  \begin{minipage}{.45\textwidth}
    \begin{center}
      \begin{tabular}{cr|ccccc}
        & &\multicolumn{5}{c}{$m$ (message)} \\
        & & 0 & 1 & 2 & 3 & 4 \\
        \cline{2-7}
        & 0 & 0 & 1 & 2 & 3 & 4 \\
        & 1 & 1 & 2 & 3 & 4 & 0 \\
        & 2 & 2 & 3 & 4 & 0 & 1 \\
        \smash{\rotatebox[origin=c]{90}{$k$ (key)}} & 3 & 3 & 4 & 0 & 1 & 2 \\
        & 4 & 4 & 0 & 1 & 2 & 3 \\
        & 5 & 0 & 1 & 2 & 3 & 4
      \end{tabular}
    \end{center}
    \end{minipage}
    \vspace{10pt}

  But $P[C=c] = \frac{6}{36} = \frac{1}{6}$. Thus, we found a pair $m \in \Ms$
  and $c \in \Cs$ for which $P[C=c|M=m] \neq P[C=c]$, so the scheme is not
  perfectly secret.
\item TODO
\end{enumerate}
\end{solution}
\begin{solution}
  By the contrapositive of theorem 2.10, if $\abs{\Ks} < \abs{\Ms}$ then the
  encryption scheme is not perfectly secret.

  Alternatively, we can see that for any $m \in \Ms$ and $c \in \Cs$ such that
  $m = c$, $P[C=c|M=m] = 0$ (since we are missing precisely the key that
  makes $c = m \oplus k = m$). However, $P[C=c] \neq 0$. The intuition for why
  key $0^{\ell}$ is no different from other keys is that $c = 0^{\ell} \oplus
  m = m$ is equivalent to $c = k' \oplus m'$, for any $k' = c \oplus m'$, and
  there is no reason why an adversary should assume that the key is
  $0^{\ell}$ instead of $k'$.
\end{solution}
\begin{solution}
  $ $
  \begin{enumerate}[label=\alph*.]
    \item \begin{align*}
    P[\priveav = 1] &= P[b=b'] \\
                    &= P[b=0] \cdot P[\text{$\Adv$ outputs 0} | b = 0] + P[b=1]
    \cdot P[\text{$\Adv$ outputs 1} | b=1] \\
    &= \frac{1}{2} \left( P[\text{$\Adv$ outputs 0}|b=0] + P[\text{$\Adv$
outputs 1}|b=1]\right)
  \end{align*}
  Then, we have:
  \[
    P[\text{$\Adv$ outputs 0} | b = 0] =
    \frac{1}{3}\cdot\underbrace{\frac{26}{26}}_{(a)} +
    \frac{1}{3}\cdot\underbrace{\frac{26}{26^2}}_{(b)} + \frac{1}{3}\cdot
    \underbrace{\frac{26^2}{26^3}}_{(c)} =
    \frac{14}{39}
\]
  where:
  \begin{enumerate}[label=\alph*.]
    \item when $\abs{k} = 1$, $\Adv$ always wins.
    \item when $\abs{k} = 2$, $\Adv$ wins when both symbols of the key equal
      each other ($k_1 = k_2$). This happens in $\frac{26}{26^2}$ keys.
    \item when $\abs{k} = 3$, $\Adv$ wins when two symbols of the key equal
      each other. This happens in $\frac{26^2}{26^3}$ keys.
  \end{enumerate}
  Also:
  \[
    P[\text{$\Adv$ outputs 1} | b = 1] =
    \frac{1}{3}\cdot\underbrace{\frac{26}{26}}_{(a)} +
    \frac{1}{3}\cdot\underbrace{\frac{26\cdot 25}{26^2}}_{(b)} + \frac{1}{3}\cdot
    \underbrace{\frac{26^2 \cdot 25}{26^3}}_{(c)} = \frac{38}{39}
\]
  where:
  \begin{enumerate}[label=\alph*.]
    \item when $\abs{k} = 1$, $\Adv$ always wins.
    \item when $\abs{k} = 2$, $\Adv$ loses when $k_1 = k_2 + 1$ (since
      $\Enc_k(abb) = c_1c_1_c_3$). This happens in 26 cases, so we care about
      the remaining $26^2-26 = 26\cdot 25$ of the $26^2$ cases.
    \item when $\abs{k} = 3$, $\Adv$ loses oncd more when the symbols of $k$ are
      consecutive, leading to $26^3-26^2 = 26^2\cdot 25$ of the $26^3$ cases.
  \end{enumerate}
  Thus, substituting in our derivation:
  \[ \frac{1}{2}\left( \frac{14}{39} + \frac{38}{39} \right) = \frac{2}{3}\]

\item TODO
\end{enumerate}
\end{solution}
\begin{solution}
  $ $
  \begin{enumerate}[label=\alph*.]
    \item The easiest proof is by Shannon's theorem: we have $\abs{\Cs} = \abs{\Ms} =
    \abs{\Ks} = 26$, each key in $\Z_{26}$ is chosen with equal probability
    ($\frac{1}{26}$), and for every $m \in \Z_{26}$ and $c \in \Z_{26}$ there is
    a unique key $k$ such that $\Enc_k(m) = c$, namely $k = c-m \pmod 26$.
    Alternatively, we can show that for each $m, c$:

    \[ P[C=c|M=m] = P[\Enc_k(m) = c] = P[K=c-m \pmod 26] = \frac{1}{26} \]

    Thus $\forall m_0, m_1 \in \Ms$ and $\forall c \in \Cs$, $P[C=c|M=m_0] =
    P[C=c|M=m_1]$, and we have perfect secrecy.
  \item By the limitation of perfect secrecy, $\abs{\Ms} \leq \abs{\Ks} = n!$,
    where $n$ is the number of symbols in the alphabet ($n = 26$ for English);
    the factorial is because that's the number of permutations on an $n$-element
    set (in particular, $\Z_{26}$). So we have an upper bound for $\abs{\Ms}$.
    Now take the set $\Ms$ of all strings of 26 characters without repeating
    any. Clearly $\abs{\Ms} = 26!$. Once more, we use Shannon's theorem:
    \begin{enumerate}[label=\arabic*.]
      \item $\abs{\Ks}  = \abs{\Ms}$, but also $\abs{\Ms} = \abs{\Cs}$ since any
        permutation on characters will map a string of 26 nonrepeated letters to another.
      \item Every key is chosen with equal probability, namely $\frac{1}{26!}$.
      \item For every $m \in \Ms$ and $c \in \Cs$, $\exists! k \in Ks$ such that
        $\Enc_k(m) = c$, since $m$ and $c$ define a unique permutation on all
        the letters of the alphabet.
    \end{enumerate}

    Therefore, the largest message space $\Ms$ for which the monoalphabetic
    cipher provides perfect secrecy is $n!$, for an $n$-element set.
  \item For an $n$-element alphabet, the Vigenère cipher using (fixed) period
    $t$ has $\abs{\Ks} = n^t$. If we encrypt messages of length $t$, then
    $\abs{\Ms} = n^t$ too. Clearly, we also have $\abs{\Cs} = n^t$.

    Again, by Shannon's theorem we see that every key is chosen with equal
    probability ($\frac{1}{n^t}$), and for each pair of plaintext and ciphertext,
    there is a unique key such that $\Enc_k(m) = c$.
  \end{enumerate}
\end{solution}
\begin{solution}
  A simple way is the following: Let $\Pi$ be a scheme satisfying definition
  2.5. Then by Lemma 2.6 $\Pi$ is perfectly secret, so by theorem 2.10,
  $\abs{\Ks} \geqslant \abs{\Ms}$. As for an $\Adv$ for which $P[\priveav = 1] >
  \frac{1}{2}$, let $\Pi$ be an arbitrary encryption scheme with $\abs{\Ks} <
  \abs{\Ms}$. 
  
  TODO finish
\end{solution}

\begin{solution}
  Let $\Ms = \set{0,1}^n$ and $\Ks = \set{0,1}^{n-t}$, with $\Enc_k(m) = [m]_{1,
  n-t} \oplus k$, i.e.\ xor the first $(n-t)$ bits of $m$ with the key $k$.
  $\Dec_k(c) = (c || 0^t) \plus (k || r)$, where $r$ is a random pad of $t$
  bits. Since we have $\frac{1}{2^t}$ chances that $r$ is precisely the missing
  part of the message, $P[\Dec_k(\Enc_k(m)) = m] = \frac{1}{2^t}$, so 
  $P[\Dec_k(\Enc_k(m)) = m] \geqslant \frac{1}{2^t}$. The perfect secrecy of
  this scheme follows from the proof of the one-time pad (this is exactly a
  one-time pad on the first $(n-t)$ bits of the message). Lower bound: $2^{n-t}
  = \abs{\Ms}\cdot 2^{-t} \leqslant \abs{\Ks}$.
\end{solution}

\section*{Chapter 3}
\setcounter{subsection}{3}
\setcounter{solution-internal}{0}

\begin{solution}
  $ $
  \begin{enumerate}
    \item Let $p$ be a positive polynomial. Since $2p$ is also a positive polynomial and
  $\negl_1$ and $\negl_2$ are negligible:

  \[ \exists N_1, N_2 \left(\forall n\geqslant N_1 \left(\negl_1(n) <
        \frac{1}{2p(n)} \right) 
\AND \forall n \geqslant N_2 \left(\negl_2(n) < \frac{1}{2p(n)} \right) \right)\]

Choose $N_3 = \max(N_1, N_2)$, then $\forall n \geqslant N_3$ we have:

\[ \negl_3(n) = \negl_1(n) + \negl_2(n) < \frac{1}{2p(n)} + \frac{1}{2p(n)} =
\frac{1}{p(n)} \]
\item Let $p, q$ be two positive polynomials. Since $p\cdot q$ is also a
  positive polynomial and $\negl_1$ is negligible:
  \[ \exists N_1 \left( \forall n \geqslant N_1 \left( \negl_1 <
  \frac{1}{q(n)p(n)} \right)\right) \]
  Then $\forall n \geqslant N_1 \left( \negl_4 = p(n)\cdot \negl_1(n) <
  \frac{1}{q(n)}\right)$.
  \end{enumerate}
\end{solution}
\begin{solution}
  Let $q(n)$ be a polynomial such that for any $k \leftarrow \Gen(1^n)$,
  $\abs{\Enc_k(0)} \leqslant q(n)$. Such a polynomial exists because the
  encryption algorithm must run in an amount of time polynomial in n. Since the
  maximum encrypted length of 0 is bounded by $q(n)$, we would like our
  adversary to choose $m_0 = 0$, and $m_1$ so that $m_1$ will always encrypt to
  a string of length greater than $q(n)$. If the adversay can to this, it
  becomes trivial to determine which message was encrypted, i.e.\ $P[\priveav =
  1] = 1$, thus the definition cannot be satisfied. Consider all strings of
  length $q(n) + 2$. % Isn't taking q(n) + 1 enough?
  Since there are $2^{q(n) + 2}$ such strings, and fewer than $2^{q(n) + 1}$
  strings of length  $\leqslant q(n)$, there must be some string $s \in
  \set{0,1}^{q(n) + 2}$ that can only encrypt to strings of length $> q(n)$. If
  the adversary chooses $m_1 = s$, then he can always win the
  indistinguishability experiment, so $\Pi$ cannot satisfy the definition, as
  desired.
\end{solution}

% Problem 3.3. See HAL2
\begin{solution}
  Let $\Pi = \left( \Gen, \Enc, \Dec \right)$ be a scheme that is secure with
  respect to the original defintion 3.8 (for messages of equal length).
  Construct a scheme $\Pi' = \left( \Gen, \Enc', \Dec'\right)$ such that:
  \begin{itemize}
    \item Given $m$, with $\abs{m} \leqslant \ell(n)$, then:
      \[ 
      \Enc'(m) = \left\{ 
        \begin{array}{ll}
          \Enc(0^{\ell - \abs{m} - 1} 1||m), & \text{if $\abs{m} < \ell(n)$} \\
          \Enc(m), & \text{if $\abs{m} = \ell(n)$}
        \end{array}
      \]
    \item $\Dec'$ applies $\Dec$ to the ciphertext, and parses the result as
      $0^t1||m$ for $t \geqslant 0$. It outputs $m$.
  \end{itemize}
  A complete answer to this exercise requires a proof showing that the existence
  of an adversary breaking $\Pi'$ with respect to the modified definitions
  implies the existence of an adversary breaking $\Pi$ with respect to
  definition 3.8.

  Informally: Given an adversary $\Adv'$ who breaks $\Pi'$, we construct an
  adversary $\Adv$ who takes the pair of plaintexts $m_0, m_1$ output by $\Adv'$
  and pads them in the same way as $\Enc'$ would. Then it outputs the padded
  messages to be encrypted. Observe that $\Adv$ outputs equal length messages,
  as required. Furthermore, if $\Adv'$ can correctly guess $b$ with probability
  greater than $\frac{1}{2}$, then this guess will also be correct for $\Adv$
  with the same probability.
\end{solution}
\begin{solution}
  Assume the scheme has indistinguishabile encryption in the presence of an
  eavesdropper (def 3.8), i.e.:

  \[ P[\priveav(n) = 1] \leqslant \frac{1}{2} + \neql(n) \] 

  TODO finish
\end{solution}

\section*{Chapter 4}
\setcounter{section}{4}
\setcounter{subsection}{4}
\setcounter{solution-internal}{0}
\begin{solution}
  TODO finish
\end{solution}
\begin{solution}
  TODO finish
\end{solution}
\begin{solution}
  TODO finish
\end{solution}
\begin{solution}
  TODO finish
\end{solution}
\begin{solution}
  TODO finish
\end{solution}
\begin{solution}
  TODO finish
\end{solution}
\begin{solution}
  Let F be a pseudorandom function. Show that each of the following MACs is insecure, even if used to authenticate fixed-length messages. In each case Gen outputs a uniform $k \in \{0, 1\}^n$. Let \langle $i$ \rangle 
  $denote an n/2-bit$
  \hspace{1mm} $encoding of the integer$ \hspace{1mm} $$i$$.
\newline
  
  1. To authenticate a message $$ m = m_1, \ldots ,m_l $$ where $$m_i \in \{0, 1\}^n$$, compute $$t := F_k(m_1) \oplus \cdots \oplus F_k(m_l)$$.
  
  2. To authenticate a message $$ m = m_1, \ldots ,m_l $$ where $$m_i \in \{0, 1\}^{n/2}$$, compute $$t := F_k(\langle 1 \rangle || m_1) \oplus \cdots \oplus F_k(\langle l \rangle || m_l)$$.
  
  3. To authenticate a message $$ m = m_1, \ldots ,m_l $$ where $$m_i \in \{0, 1\}^{n/2}$$, choose uniform $$r \leftarrow \{0, 1\}^n$$, compute $$t := F_k(r) \oplus F_k(\langle 1 \rangle || m_1) \oplus \cdots \oplus F_k(\langle l \rangle || m_l)$$, and let the tag be $$\langle r, t \rangle $$

\end{solution}
\begin{solution}
\noindent
   Let F be a pseudorandom function. Show that each of the following MACs is insecure, even if used to authenticate fixed-length messages.In each case Gen outputs a uniform $k \in \{0, 1\}^n$. Let \langle $i$ \rangle 
  $denote an n/2-bit$
  \newline
  $encoding of the integer$ \hspace{1mm} $$i$$.
\newline

1. To authenticate a message $$ m = m_1, \ldots ,m_l $$ where $$m_i \in \{0, 1\}^n$$, compute $$t := F_k(m_1) \oplus \cdots \oplus F_k(m_l)$$.
2. To authenticate a message $$ m = m_1, \ldots ,m_l $$ where $$m_i \in \{0, 1\}^{n/2}$$, compute $$t := F_k(\langle 1 \rangle || m_1) \oplus \cdots \oplus F_k(\langle l \rangle || m_l)$$.
3. To authenticate a message $$ m = m_1, \ldots ,m_l $$ where $$m_i \in \{0, 1\}^{n/2}$$, choose uniform $$r \leftarrow \{0, 1\}^n$$, compute $$t := F_k(r) \oplus F_k(\langle 1 \rangle || m_1) \oplus \cdots \oplus F_k(\langle l \rangle || m_l)$$, and let the tag be $$\langle r, t \rangle $$

Answer :

* Part 1 :

Reorder the blocks in "m" and the tag doesn't change.

* Part 2 : 

Query

* $$m^1 = m_1 || m_2$$, tag $$t_1 = F_k(\langle 1 \rangle || m_1) \oplus F_k(\langle 2 \rangle || m_2) $$
* $$m^2 = m_3 || m_2$$, tag $$t_2 = F_k(\langle 1 \rangle || m_3) \oplus F_k(\langle 2 \rangle || m_2) $$
* $$m^3 = m_3 || m_4$$, tag $$t_3 = F_k(\langle 1 \rangle || m_3) \oplus F_k(\langle 2 \rangle || m_4) $$

Thus $$ m^* = m^1 \oplus m^2 \oplus m^3 = m_1 || m_4$$, tag $$t = t_1 \oplus t_2 \oplus t_3 = F_k(\langle 1 \rangle || m_1) \oplus F_k(\langle 2 \rangle || m_4) $$.
\newline
* $$Part 3 :$$
\newline
Let $$m \in \{0, 1\}^{n/2}$$. When choosing $$r = \langle 1 \rangle || m$$, $$t = F_k(r) \oplus F_k(\langle 1 \rangle || m) = 0^n$$.

Thus $$ t = \left\langle \langle 1 \rangle || m, 0^n \right\rangle $$ will be a valid tag for "m".
\end{solution}
\begin{solution}
  Let "F" be a pseudorandom function. Show that the following MAC for messages of length "2n" is insecure: Gen outputs a uniform $$k \in \{0, 1\}^n$$. To authenticate a message $$m_1 || m_2$$ with $$|m_1| = |m_2| = n$$, compute the tag $$F_k(m_1) || F_k(F_k(m_2))$$.
  
  Answer: 
  
  Query

* $$m^1=m^*_1||m^*_1$$, $$t^1= t^1_1||t^1_2 = F_k(m^*_1) || F_k(F_k(m^*_1)) $$
* $$m^2=m^*_2||m^*_2$$, $$t^2= t^2_1 || t^2_2 = F_k(m^*_2) || F_k(F_k(m^*_2)) $$

Hence for $$m^*=m^*_1||m^*_2$$, $$t^* = t^1_1||t^2_2$$


\end{solution}
\begin{solution}
  TODO finish
\end{solution}
\begin{solution}
  TODO finish
\end{solution}
\begin{solution}
  TODO finish
\end{solution}
\begin{solution}
  TODO finish
\end{solution}
\begin{solution}
  Prove that the following modifications of basic CBC-MAC do not yield a secure MAC (\,even for fixed-length messages)\,:

1. Mac outputs all blocks $$t_1, \ldots , t_l$$rather than just $$t_l$$. (\,Verification only checks whether $t_l$ is correct.)\,

2. A random initial block is used each time a message is authenticated. That is, choose uniform $$t \in \{0, 1\}^n$$, run basic CBC-MAC over the “message” $$t_0,m_1, \ldots ,m_l$$, and output the tag $$ \langle t_0, t_l \rangle$$. Verification is done in the natural way.

The Answer :

* Part 1:

Query

* $$m^1 = B_0 || B_1$$, $$t^1 = t_0 || t_1$$
* $$m^2 = B_2 || B_3$$, $$t^2 = t_2||t_3$$

We know $$F_k(B_0) = t_0$$ and $$F_k(B_2) = t_2$$. Hence 

$$
MAC_k(B_0 || B^*_2) = F_k(B_0) || F_k(F_k(B_0) \oplus B^*_2) = t_0 || F_k(t_0 \oplus B^*_2)
$$

Let $$t_0 \oplus B^*_2 = B_2$$, i.e., $$B^*_2 = t_0 \oplus B_2$$. Then

$$
MAC_k(B_0 || t_0 \oplus B_2) = t_0 || F_k(t_0 \oplus t_0 \oplus B_2) = t_0 || F_k(B_2) = t_0 || t_2
$$

Therefore, $$\langle B_0 || t_0 \oplus B_2, t_0 || t_2 \rangle$$ is a valid pair of message and tag.

* Part 2:

Query

* $$m^1 = B_0 || B_1$$, $$t^1 = \langle r_1, t_1 \rangle$$
* $$ m^2 = B_2 || B_3 $$, $$ t^2 = \langle r_2, t_2 \rangle $$

Hence for $$m^* = B_0 || B_1 || t_2 \oplus r_2 || B_2 || B_3 $$, $$ t^* = \langle r, t_2 \rangle$$ should be a valid tag.

\end{solution}
\begin{solution}
  Show that appending the message length to the end of the message before applying basic CBC-MAC does not result in a secure MAC for arbitrary-length messages.
  
  The Answer : 
  
  Query

* $$m_1 = B_0 || B_1$$, $$t_1 = MAC_k(m_1 || \langle |m_1| \rangle)$$
* $$m^*_1 = B^*_0 || B^*_1$$,  $$t^*_1 = MAC_k(m^*_1 || \langle |m^*_1| \rangle)$$
  * $$ |m^*_1| = |m_1|$$
* $$m_2 = m_1 || \langle |m_1| \rangle || B_2 || B_3$$, $$t_2 = MAC(m_2 || \langle |m_2| \rangle)$$

To be specific, the process of computing $$t_2$$ for message $$m_2$$ is listed below:

* $$c_0=F_k(B_0)$$
* $$c_1=F_k(c_0 \oplus B_1)$$
* $$ t_1=F_k(c_1 \oplus \langle |m_1| \rangle) $$
* $$ c_3=F_k(t_1 \oplus B_2) $$
* $$ c_4=F_k(c_3 \oplus B_3) $$
* $$ t=F_k(c_4 \oplus \langle | m_2 | \rangle) $$

Hence, if we change $$m_1$$ to $$ m^*_1 $$,

* $$c^*_0=F_k(B^*_0)$$
* $$c^*_1=F_k(c^*_0 \oplus B^*_1)$$
* $$ t^*_1=F_k(c^*_1 \oplus \langle |m^*_1| \rangle) $$

In order to keep the result of MAC, it must hold that $$ t_1 \oplus B_2 = t_1^* \oplus B^*_2$$. Thus

$$
B^*_2 = t_1 \oplus B_2 \oplus  t_1^*
$$

Therefore

* $$ c^*_3=F_k(t^*_1 \oplus B^*_2) = F_k(t^*_1 \oplus  t_1 \oplus B_2 \oplus  t_1^*) = F_k(t_1 \oplus B_2) = c_3$$
* $$ c^*_4 = F_k(c^*_3 \oplus B_3) = F_k(c_3 \oplus B_3) = c_4$$
* $$ t^* = F_k(c^*_4 \oplus \langle | m^*_2 | \rangle)  = F_k(c_4 \oplus \langle | m_2 | \rangle) =t$$
  * $$ |m^*_2|=|m_2|$$ can be easily get since $$ |m^*_1| = |m_1| $$

Hence we get a message and its valid tag $$\langle m^*, t^* \rangle $$ where

$$
m^* := m^*_1 || \langle | m^*_1| \rangle || t_1 \oplus B_2 \oplus  t_1^* || B_3 \\
t^* = t
$$

\end{solution}
\begin{solution}
  Show two types of forgery attacks for authenticated encryption scheme CBC-XOR.
  
  Given a pseudorandom permutation F

  $Gen: k \ll {0, 1}^n$

  Enc: On input a message 
  m = B_0 || B_1 || ... || B_l \hspace{1mm} 
  $and a key k, uniformly generate an IV$  \ll ${0, 1}^m$
    
    1. Compute B_{l+1} = B_0 || B_1 || ... || B_l
    
    2. Do CBC encryption on m || B_{l+1} $using k and IV$
        
        - Output ciphertext c := IV || c_0 || c_1 || ... || c_l || c_{l+1}
        

  Dec: On input a ciphertext c = IV || c_0 || c_1 || ... || c_l || c_{l+1} $and a key k$
   
    1. Do CBC decryption on c_0 || c_1 || ... || c_l || c_{l+1} $using k and IV$
    
    2. Check if B_{l+1} = B_0 || B_1 || ... || B_l
        - $If true, output plaintext$ \hspace{1mm} B_0 || B_1 || ... || B_l
        - $If false, output error$

\newline
  Answers :
\newline
  Method 1 - Truncation

  Query $$ m = B_0 || B_1 || (B_0 \oplus B_1) $$ and obtain the ciphertext $$ c = IV || c_0 || c_1 || c_2 || c_3 $$.

  Thus $$ c^* = IV || c_0 || c_1 || c_2 $$ should be a valid ciphertext for $$m^* = B_0 || B_1$$

  Method 2 - Swap

  Query $$m = B_0 || B_1 || B_2 $$ and obtain the ciphertext $$c = IV || c_0 || c_1 || c_2 || c_3 $$

  Thus

* $$F_k(IV \oplus B_0) = c_0$$
* $$F_k(c_0 \oplus B_1) = c_1$$
* $$ F_k(c_1 \oplus B_2) = c_2 $$
* $$ F_k(c_2 \oplus B_0 \oplus B_1 \oplus B_2) = c_3 $$

Hence $$c^* = IV || c_1 || c_0 || c_2 || c_3$$ should be a valid tag for $$m^* = B^*_1 || B^*_0 || B^*_2$$, where

* $$ B^*_0 = c_0 \oplus B_1 \oplus IV$$
* $$ B^*_1 = IV \oplus B_0 \oplus c_1$$
* $$ B^*_2 = c_1 \oplus B_2 \oplus c_0$$
* $$ B^*_0 \oplus B^*_1 \oplus B^*_2 =  c_0 \oplus B_1 \oplus IV \oplus IV \oplus B_0 \oplus c_1 \oplus c_1 \oplus B_2 \oplus c_0 = B_0 \oplus B_1 \oplus B_2$$
\end{solution}
\begin{solution}
  TODO finish
\end{solution}
\section*{Chapter 5}
\setcounter{section}{5}
\setcounter{subsection}{5}
\setcounter{solution-internal}{0}
\begin{solution}
  TODO
\end{solution}
\begin{solution}
  TODO
\end{solution}

\begin{solution}
  TODO
\end{solution}

\begin{solution}
  TODO
\end{solution}

\begin{solution}
 Problem

Let \(Gen, $$H$$\) be a collision-resistant hash function. Is \(Gen, $$\hat H$$\) defined by$$\hat H ^s (x) \overset{def}{=} H^s(H^s(x))$$necessarily collision resistant?

* Solution

Assuming that $\hat H$ is not collision-resistent, i.e. 

$$
\exists x\neq y, \hat H^s(x) = \hat H^s(y)
$$

Thus $$H^s(H^s(x)) = H^s(H^s(y)) $$

* If $$ H^s(x) = H^s(y)$$, $$(x,y)$$ is a pair of collision for $$H$$
* If $$ H^s(x) \neq H^s(y)$$, let $$x'=H^s(x)$$, $$y'=H^s(y)$$. 
  * $$H^s(H^s(x)) = H^s(H^s(y)) $$, $$(x',y')$$ is a pair of collision for $$H$$

Therefore, $\hat H$ is not collision-resistent implies $H$ is not collision-resistent. Then $H$ is collision-resistent implies $\hat H$ is collision-resistent.

\end{solution}

\begin{solution}
  TODO
\end{solution}

\begin{solution}
  TODO
\end{solution}



\end{document}